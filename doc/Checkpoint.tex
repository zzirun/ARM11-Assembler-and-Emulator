\documentclass[11pt]{article}

\usepackage{fullpage}
\usepackage[dvipsnames]{xcolor}

\begin{document}

\title{ARM Checkpoint}
\author{Ivy Tam, Wei Yi Tee, Zirun Zhai, Soon Zhi Ho (Brandon)}

\maketitle

\section{Group Organisation}

After reading through the specification, we had a discussion to resolve any uncertainties to make sure we reached a common understanding of what the project entails. Once we were confident with our design choices, we worked on implementing the emulator framework together which required creating a binary file loader, defining data structures for memory, registers, machine state and instructions. We then delegated tasks to each member to maximize efficiency and minimise merge conflicts. Wei Yi worked on the Data Processing Instructions, Ivy implemented the Multiply Instructions, Zirun worked on the Single Data Transfer Instructions, while Brandon completed the Branch Instructions and decoder. We continued to communicate closely despite working individually to make sure everyone was clear of our progress and to reduce code repetition where functions could be reused. \par 
Following this, we completed the emulator loop together and debugged the code according to the test cases. Throughout the process, we demonstrated great synergy when we faced one particular challenge where our program would timeout in one of the test cases - loop 01, which had about 4 million iterations. By profiling, we identified that our decoder was inefficient due to frequent calls to a bit extraction function which computed masks with \emph{pow}. Our solution was to hard-code the appropriate masks and apply bitwise operations to extract the bits instead. This improved efficiency tremendously at the cost of lower extensibility.\par 
Finally, we cleaned up the code by reorganising the files and code structure. We made sure that our code had consistent style. We also minimized the presence of magic numbers by defining constants and used type aliases to improve readability. \par
Our group has worked well for the tasks thus far. We overcame the constraint of not being able to meet up physically through strong and constant communication via Whatsapp messaging or Discord calls. Our emphasis on clear, readable coding styles and relevant comments made it very easy for members to read and edit each other’s work. We adapted fluidly to both collaborating on a task and individually working on separate parts. However, we could learn to utilize git pull requests in the future as suggested by our mentor to coordinate the pushing of code and reduce time wasted on resolving unnecessary merge conflicts. We also tend to be indecisive when making design decisions together. We will overcome this in future tasks, as we have realised that there is no right or wrong implementation - it boils down to design choice. \par
\break
\section{Implementation Strategies}

We decided to split the \textbf{emulator} into 5 main files :

\begin{itemize}
	\item \textcolor{blue}{\textbf{\emph{emulate.c}}} : The main function of the emulator.
	\begin{itemize}
		\item We initialised the struct \emph{machine\_state} (stores registers, memory, instructions and processor state) on the heap because we want it to be persistent throughout the program as it will be 			  accessed and modified by most function calls. Alternatively, we could have stored the struct in the stack frame of the main function, but we ultimately decided against this as it increases the risk 			  of stack overflow.
		\item Calls binary file loader function to handle parsing of the input argument and loading of instructions into memory in little endian format.
		\item Simulates the ARM pipeline in the form of a loop which will fetch, decode and execute the instructions by calling the appropriate functions.
	\end{itemize}
	\item \textcolor{blue}{\textbf{\emph{decode.c}}} : Contains the decode function translating binary instructions into a decoded instruction struct, making the instruction fields more accessible during the 									     			   execution stage of the pipeline.
	\item \textcolor{blue}{\textbf{\emph{execute.c}}} : Contains the execute function which first checks the condition codes of the instruction, then calls the execution function according to its instruction 														type. 
	\item \textcolor{blue}{\textbf{\emph{types.h}}} : Contains definitions of various types we will use throughout our project such as constants, enums and structs which model the state of the machine and 														  instructions.
	\begin{itemize}
		\item We defined a struct - \emph{decoded\_instr} which contains the fields of an instruction according to its type (through a union)  such as its opcodes, destination register, offset etc. We will be 				  able to reuse this struct as a ‘stepping stone’ when encoding assembly code to binary code in the assembler.
	\end{itemize}
	\item \textcolor{blue}{\textbf{\emph{utils.c}}} : Contains some generic functions to avoid code repetition and aid in debugging.
	\begin{itemize}
	\item For example, both Single Data Transfer and Data Processing instructions involved a similar process of extracting operand 2 as a shifted register. Hence, we implemented helper functions in utils used in 		  both functions.
	\end{itemize}
\end{itemize}

\begin{flushleft}
It is likely that we will reuse the constants and data types from our emulator (eg. instruction struct to help encode from assembly to binary) while handling the assembler. We plan on implementing two-pass assembly, as single-pass assembly appears to give rise to further complications as described in the specification. There are certain tasks we think may be challenging including designing a parser for text-based input which is different from the emulator, consideration of which data structure to use when constructing the symbol table during the first pass, writing into a binary file as the output and conversion between endianness. We plan to mitigate these challenges by having in-depth discussions to dissect the details before implementation, instead of having a draft then refine repeatedly. We have yet to come up with a topic for our extension. However, we plan to draw from the given theme and brainstorm as a team with our mentor for inspiration to come up with an original topic. We have also decided to focus on the upcoming C Programming Test over the next few days before continuing with the project. 
\end{flushleft}

\end{document}
